% !TEX encoding = UTF-8
% !TEX program = pdflatex
% !TeX spellcheck = en_GB
% !BIB = biber


\documentclass[english]{article}
\usepackage{babel}
\usepackage[utf8]{inputenc}
\usepackage{graphicx}
\graphicspath{{./images/}}
\usepackage{hyperref}
\usepackage{csquotes}
\hypersetup{
    colorlinks=true, 
	linkcolor=blue, 
	filecolor=blue, 
	citecolor = black,       
	urlcolor=blue, 
}


\usepackage{biblatex}
\addbibresource{bib.bib}

\title{5G Security}
\author{Alessandro Castelli \\ ID:\@147073 \\ E-mail: castelli.alessandro@spes.uniud.it}

\begin{document}

\maketitle
5
\begin{abstract}
	Negli ultimi decenni, le comunicazioni wireless hanno subito una rapida
	evoluzione, alimentata dalla crescente domanda degli utenti per connessioni
	sempre più veloci, affidabili e performanti. Tra le innovazioni tecnologiche
	più rilevanti, la tecnologia 5G si è affermata come la nuova frontiera delle
	telecomunicazioni mobili, promettendo una capacità di banda superiore,
	latenze ridotte e una enorme densità di connessioni per dispositivi
	intelligenti e IoT. Tuttavia, insieme a queste straordinarie capacità,
	emergono anche nuove sfide in termini di sicurezza. Il 5G introduce una
	infrastruttura di rete più complessa e decentralizzata, aumentando il rischio
	di vulnerabilità e attacchi informatici. Questo articolo esamina i principali
	aspetti della sicurezza del 5G, evidenziando le vulnerabilità della rete e
	discutendo le soluzioni tecnologiche avanzate sviluppate per mitigare i rischi.
\end{abstract}

\clearpage

\tableofcontents
\newpage
\section{Introduzione}

La storia della comunicazione mobile è un viaggio di innovazione continua.
Tutto inizia con il 1G~\cite{dangi2021study}, negli anni '70 e '80, quando la
trasmissione dei dati era analogica. Era l'inizio, ma con grandi limiti: la
qualità era bassa, la sicurezza inesistente e le chiamate potevano essere
facilmente intercettate. Tuttavia, consentiva qualcosa di rivoluzionario per
l'epoca: la connettività mobile e i primi servizi vocali, anche se in modo
rudimentale.

Con l'arrivo del 2G nel 1991, la situazione cambiò drasticamente. La
comunicazione diventò digitale, affrontando molti dei problemi del 1G. Ora
c'era più sicurezza, efficienza e una maggiore larghezza di banda. Si aprirono
così le porte a nuovi servizi come i messaggi di testo (SMS), rendendo la
comunicazione mobile più sofisticata.

Successivamente, il 3G introdusse il concetto di banda larga mobile. Non era
più solo una questione di chiamate o messaggi, ma anche di videochiamate,
navigazione su Internet e una trasmissione dati molto più veloce. Il 3G
rappresentava una svolta, anche se soffriva ancora di problemi legati allo
spettro e alla latenza, dimostrando che la tecnologia aveva ancora margini di
miglioramento.

Quando arrivò il 4G, tra il 2009 e il 2010, il mondo della comunicazione mobile
fece un enorme balzo in avanti. Grazie a tecnologie come LTE, la velocità dei
dati aumentò significativamente, portando il mobile streaming, i giochi online
e i video in alta definizione a portata di mano ovunque. Le velocità di
download teoriche potevano raggiungere i 100 Mbps, sebbene nella pratica
fossero spesso inferiori.

Infine, con l'avvento del 5G, si entrò in una nuova era. Non si parlava più
solo di miglioramenti incrementali, ma di una rivoluzione. Il 5G promette
velocità fino a 20 Gbps~\cite{javid20225g}, una latenza ultra bassa e la
capacità di connettere miliardi di dispositivi contemporaneamente. È una
tecnologia pensata per un mondo interconnesso, dove non ci sono solo
smartphone, ma anche automobili, dispositivi IoT, smart cities e sistemi
industriali automatizzati. Il 5G è il fondamento di quella che sarà l'Internet
of Everything, dove ogni aspetto della vita quotidiana è connesso e integrato
in una rete globale.

Il \textbf{5G} rappresenta quindi l'ultima evoluzione nella tecnologia di
comunicazione mobile, portando miglioramenti significativi come
\textit{larghezza di banda elevata} e \textit{latenza estremamente bassa}.
Questa tecnologia supporta applicazioni avanzate come la realtà aumentata (AR),
la realtà virtuale (VR) e le comunicazioni ultra-affidabili a bassa latenza
(URLLC). Nel marzo 2018, il
\href{https://www.3gpp.org/technologies/5g-system-overview}{3GPP} ha rilasciato
la 15ª release degli standard di comunicazione mobile, stabilendo le basi per
il \textbf{5G}. La velocità di trasmissione di questa nuova tecnologia consente
agli utenti di usufruire di trasferimenti dati notevolmente superiori, in
particolare per applicazioni che richiedono un alto throughput, come lo
streaming video in alta definizione~\cite{javid20225g}.

La riduzione della latenza è un altro obiettivo chiave, con la previsione di
una latenza inferiore a 1 millisecondo, aprendo la strada all'uso in tempo
reale di applicazioni critiche come la telemedicina e la guida autonoma.
Inoltre, il 5G permette la connessione simultanea di un numero molto maggiore
di dispositivi rispetto alle generazioni precedenti, una caratteristica
fondamentale vista la continua crescita del mercato IoT.

Grazie a tutto questo, il 5G diventerà una base per una rete che connette non
solo persone, ma anche oggetti, dispositivi e macchine. Si sta parlando,
quindi, di un sistema che sta rivoluzionando il modo in cui immaginiamo
internet, non più solo come uno scambio di dati tra persone, ma come
un'integrazione massiccia tra esseri umani e macchine, e tra macchine stesse.

E qui entrano in gioco i tre scenari fondamentali del 5G:\@ \textbf{eMBB, mMTC
	e uRLLC}. Ognuno di questi rappresenta una sfaccettatura dell'intera visione
del 5G~\cite{Ji2018Overview}: \textbf{eMBB} è pensato per una banda larga
mobile potenziata, consentendo download rapidissimi e streaming ad altissima
qualità; \textbf{mMTC} è rivolto alla comunicazione massiva tra macchine,
essenziale per supportare l’IoT (Internet of Things); \textbf{uRLLC} invece è
cruciale per applicazioni che richiedono una latenza minima e un'affidabilità
estrema, come la telechirurgia o i veicoli autonomi. Ma quali sono le
tecnologie che effettivamente rendono possibile tutto questo?

Il 3GPP ha definito più di 70 tipi di file 5G SA1 necessari a questo scopo, e
le tecnologie chiave sviluppate per il 5G includono:
\textbf{\hyperlink{MIMO}{Massive MIMO}}, \textbf{\hyperlink{FBMC}{filter bank
		based multicarrier (FBMC)}}, \textbf{\hyperlink{FullDuplex}{Full Duplex}},
\textbf{\hyperlink{UDN}{Ultra Dense networking (UDN)}},
\textbf{\hyperlink{SDN}{software-defined networking (SDN)}} e
\textbf{\hyperlink{NFV}{network function virtualization
		(NFV)}}~\cite{Ji2018Overview}.
%%%%%%%%%%%%%%%%%%%%%%%%%%%%%%%%%%%%%%%%%%%%%%%%%%%%%%%%%%%%%%%%%%%%%%%%%%%%%%%% 
\section{Threat model}
In letteratura sono state identificate numerose sfide legate alla sicurezza del
5G. Gli asset principali coinvolti includono i dati sensibili degli utenti,
come informazioni personali e dati di navigazione, che transitano attraverso la
rete 5G. A questo si aggiungono l'infrastruttura di rete stessa, composta da
stazioni base, server e dispositivi di rete, nonché i dispositivi degli utenti
finali, tra cui smartphone, tablet e dispositivi IoT, che rappresentano una
parte essenziale del sistema. È importante notare che, oltre ai dispositivi
finali, i sistemi di controllo della rete, incaricati della gestione del
traffico e dell'autenticazione, necessitano di una protezione adeguata per
garantire l'integrità e la disponibilità della rete.

Le reti 5G introducono diverse tecnologie innovative, come il
\textit{\hyperlink{SDN}{software-defined networking (SDN)}} e il
\textit{\hyperlink{NFV}{network function virtualization (NFV)}}, che offrono
vantaggi significativi in termini di flessibilità e scalabilità. Tuttavia,
queste tecnologie espongono anche la rete a nuovi rischi di sicurezza, come
l'esaurimento delle risorse e vulnerabilità nelle interfacce di programmazione,
che possono diventare obiettivi per attacchi mirati. Inoltre, il
\textit{\hyperlink{MIMO}{Massive Multiple-Input Multiple-Output}} (MIMO) e le
\textit{\hyperlink{mmWave}{comunicazioni a onde millimetriche (mmWave)}}
aumentano la capacità delle reti, ma devono affrontare problemi di sicurezza
legati alla gestione delle risorse e alla segretezza delle informazioni. Anche
il \textit{cloud computing} e il \textit{Multi-access Edge Computing} (MEC)
possono contribuire a migliorare l'efficienza della rete, ma l'archiviazione e
l'elaborazione dei dati nel cloud aumentano il rischio di attacchi ai dati
sensibili~\cite{Ahmad2019Security}.

I potenziali attaccanti in questo contesto possono variare notevolmente. Da una
parte, ci sono hacker individuali che cercano di intercettare o manipolare i
dati per scopi illeciti. Dall'altra, vi sono gruppi di cyber-criminali
organizzati, e in alcuni casi anche attori sponsorizzati da stati, il cui
obiettivo potrebbe essere ottenere accesso non autorizzato a informazioni
sensibili o causare danni alla rete per ragioni economiche o geopolitiche. A
rendere più complesso il quadro ci sono anche malintenzionati interni, ovvero
personale con accesso legittimo alla rete, che potrebbe abusare delle proprie
autorizzazioni. In questo contesto, è cruciale riconoscere che le vulnerabilità
non provengono solo dall'esterno, ma anche da configurazioni errate e pratiche
di sicurezza inadeguate da parte del personale autorizzato. I vettori di
attacco attraverso cui questi attori possono colpire sono molteplici. La
sicurezza delle interfacce radio, potrebbe essere compromessa se tali
protocolli non vengono implementati correttamente o se si sfruttano
vulnerabilità complesse. Questo apre la strada ad attacchi come
l'intercettazione delle comunicazioni o i classici man-in-the-middle. Anche
l'integrità del piano utente rappresenta un punto critico: nonostante
l'introduzione di crittografia avanzata nel 5G, attacchi mirati potrebbero
sfruttare eventuali falle nel processo di protezione dei dati, compromettendo
così l'integrità dei dati trasmessi senza essere rilevati.

Durante il roaming, un altro momento delicato per la sicurezza, i parametri di
protezione degli utenti potrebbero non essere aggiornati correttamente al
passaggio tra reti diverse, esponendo gli utenti a potenziali attacchi di
intercettazione o manipolazione dei dati. Questa problematica evidenzia la
necessità di un'adeguata sincronizzazione e aggiornamento delle misure di
sicurezza tra diverse reti, al fine di garantire una protezione continua e
coerente.

L'infrastruttura del 5G, inoltre, non è immune da attacchi DoS (Denial of
Service). Nonostante i progressi nelle misure di protezione, i sistemi di
controllo della rete rimangono visibili e possono essere vulnerabili a
interruzioni del servizio, soprattutto se i canali di controllo non sono
crittografati adeguatamente. Qui, la gestione delle credenziali e delle
configurazioni diventa fondamentale per prevenire accessi non autorizzati.

Infine, un importante vettore di attacco è rappresentato dai dispositivi degli
utenti finali. Spesso, questi dispositivi non sono dotati di misure di
sicurezza sufficienti a livello di sistema operativo e applicazioni, rendendoli
vulnerabili a malware, attacchi DoS o manipolazioni dei dati di configurazione,
compromettendo così l'intera rete. Questa vulnerabilità è accentuata dalla
crescente diffusione dei dispositivi IoT, che possono presentare vulnerabilità
intrinseche dovute a progettazioni inadeguate o a una mancanza di aggiornamenti
regolari.

In conclusione, il panorama delle minacce nella rete 5G è complesso e
variegato, richiedendo un approccio multifattoriale per la protezione degli
asset e la mitigazione dei rischi. Un'attenzione particolare deve essere
dedicata non solo alle tecnologie e ai protocolli di sicurezza, ma anche alla
formazione continua del personale e alla consapevolezza degli utenti finali
riguardo ai rischi e alle migliori pratiche per la sicurezza.
%%%%%%%%%%%%%%%%%%%%%%%%%%%%%%%%%%%%%%%%%%%%%%%%%%%%%%%%%%%%%%%%%%%%%%%%%
\section{Security goals}
Describe the security goals that we aim to achieve on the identified assets,
according to the CIAAA model. \\ Ogni sistema di comunicazione affinchè sa
considerato sicuro ha bisogno di protocolli e tecnolgie di protezioneche
proteggano le risorse usate la comunizazione in modo tale da garantire che i
dati rispettino le proprietà di: \textbf{Riservatezza}, \textbf{Integrità},
\textbf{Disponibilità}, \textbf{Autenticità,
	\textbf{Accountability}}~\cite{mohan2022cyber}. \\ Questi obiettivi sono
essenziali per garantire che le reti 5G possano operare in modo sicuro e
affidabile, proteggendo i dati e le comunicazioni degli utenti. \\ Iniziamo con
la Riservatezza. Questo obiettivo mira a garantire che solo gli utenti
autorizzati possano accedere alle informazioni sensibili. Nelle reti 5G, dove
la quantità di dati trasmessi è enorme e include informazioni personali e
aziendali, è cruciale implementare misure di crittografia e autenticazione
robusta. La riservatezza non è solo una questione di protezione dei dati, ma
anche di fiducia degli utenti nel sistema. Se gli utenti percepiscono che i
loro dati non sono al sicuro, potrebbero essere riluttanti a utilizzare i
servizi offerti. \\ Passando all'Integrità, questo obiettivo si concentra sulla
protezione dei dati da modifiche non autorizzate. In un ambiente 5G, dove i
dispositivi IoT e le applicazioni critiche sono sempre più interconnessi, è
fondamentale garantire che le informazioni rimangano accurate e non vengano
alterate durante la trasmissione. \\ La Disponibilità è un altro obiettivo
chiave. Le reti 5G devono essere sempre disponibili per garantire che gli
utenti possano accedere ai servizi in qualsiasi momento. Questo è
particolarmente importante per applicazioni critiche come quelle nel settore
sanitario o nei trasporti. Per raggiungere questo obiettivo, è necessario
implementare soluzioni di ridondanza e resilienza, in modo da mitigare gli
effetti di attacchi come i Distributed Denial of Service (DDoS), che possono
compromettere la disponibilità dei servizi. \\ L'Autenticità è fondamentale per
garantire che le entità coinvolte nelle comunicazioni siano chi dichiarano di
essere. In un contesto 5G, dove le interazioni avvengono tra una varietà di
dispositivi e utenti, è essenziale implementare meccanismi di autenticazione
robusti. Ciò non solo protegge gli asset da accessi non autorizzati, ma
contribuisce anche a costruire un ecosistema di fiducia tra i vari attori
coinvolti. \\ Infine, la Accountability è cruciale per garantire che tutte le
azioni e le transazioni all'interno della rete possano essere tracciate e
verificate. Questo obiettivo è particolarmente importante per la conformità
alle normative e per la gestione delle responsabilità. Implementare sistemi di
logging e monitoraggio consente di rilevare attività sospette e di rispondere
rapidamente a potenziali incidenti di sicurezza. \\ In sintesi, il modello
CIAAA fornisce un framework completo per affrontare le sfide di sicurezza nelle
reti 5G. Ogni obiettivo è interconnesso e contribuisce a creare un ambiente
sicuro e affidabile per gli utenti e le applicazioni. La protezione degli asset
in un contesto 5G richiede un approccio olistico che integri tecnologie
avanzate, politiche di sicurezza rigorose e una continua vigilanza per
affrontare le minacce emergenti.
%--------------------------------------------------------------------------------
\section{Security service and implementation}
Describe the security services to be implemented, and how these are
implemented: protocols, algorithms, procedures, etc.

L'architettura del 5G è organizzata attraverso 3 strati: uno strato applicatvo,
uno strato di servizio e uno strato di trasporto~\cite{Jover2018Security}. Ogni
strato è progettato con specifiche funzionalità di sicurezza che, combinate tra
loro, creano un sistema sicuro e resistente alle minacce. I principali elementi
di sicurezza nel 5G includono:

\begin{itemize}
	\item \textbf{Sicurezza dell'accesso alla rete}: Meccanismi che permettono a
	      un dispositivo utente (UE) di autenticarsi e accedere in modo sicuro ai
	      servizi di rete. L'UE scambia messaggi di protocollo attraverso la rete di accesso
	      con la rete di servizio (SN).
	      Le chiavi crittografiche sono memorizzate nel modulo \texttt{USIM} del dispositivo e nell'ambiente
	      dell'operatore. Questo garantisce la sicurezza dei dati e delle comunicazioni,
	      impedendo accessi non autorizzati.
	\item \textbf{Sicurezza del dominio di rete}: Un insieme di caratteristiche che permettono
	      ai nodi della rete di scambiarsi in modo sicuro i dati del piano di controllo e del piano
	      utente all'interno delle reti 3GPP e tra reti diverse. Le tecniche di protezione includono
	      crittografia e integrità per evitare intercettazioni e manomissioni.
	\item \textbf{Sicurezza del dominio utente}: Si concentra sulla protezione del dispositivo
	      dell'utente e dei dati contenuti, impedendo l'accesso non autorizzato al terminale mobile.
	      Sono implementati meccanismi hardware per proteggere il modulo   \texttt{USIM} e prevenire la
	      manomissione dei terminali, garantendo l'autenticità dell'utente.
	\item \textbf{Sicurezza del dominio dell'architettura basata su servizi}: Protegge la
	      registrazione, la scoperta e l'autorizzazione degli elementi di rete, nonché le interfacce
	      basate su servizi. Consente l'integrazione sicura delle nuove funzioni di rete virtuali
	      del 5G, e supporta il roaming sicuro, coinvolgendo la rete di servizio
	      e la rete domestica.
	\item \textbf{Visibilità e configurabilità della sicurezza}: Consente agli utenti di essere
	      informati sulla presenza di funzioni di sicurezza e offre la possibilità di configurare le
	      caratteristiche di sicurezza in base alle esigenze. Le specifiche di sicurezza del 5G,
	      definite dal 3GPP, stabiliscono funzionalità opzionali, fornendo gradi di libertà per
	      l'implementazione e il funzionamento sicuro della rete.
\end{itemize}

Le procedure di sicurezza del 5G si basano su un framework a derivazione
gerarchica. La chiave a lungo termine \texttt{K} è conservata dalla
\texttt{Authentication Credential Repository and Processing Function} (ARPF)
mentre la USIM conserva la copia corrispondente di tale chiave simmetrica
dell'utente~\cite{Jover2018Security}. Tutte le altre chiavi sono derivate da
essa (per vedere come vengono generate le chiavi guarda 13).

Il 3GPP ha introdotto l'\texttt{Extensible Authentication Protocol (EAP)} per
l'Autenticazione e l'Accordo delle Chiavi, definendo
l'\texttt{\hyperlink{EAP-AKA}{EAP-AKA}} e il \texttt{\hyperlink{5G AKA}{5G
		AKA}} come metodi obbligatori di autenticazione per i dispositivi (UE) e la
rete. Questi protocolli garantiscono un'autenticazione reciproca tra il
dispositivo e la rete, oltre a proteggere la sicurezza e la cifratura dei
servizi. Durante la registrazione, un dispositivo 5G invia il **SUCI** per
avviare il processo di autenticazione basato sul protocollo selezionato.

Le specifiche di sicurezza del 5G definiscono vari contesti di sicurezza per
diverse situazioni: per una singola rete di servizio 5G (SN), tra più SN e tra
reti 5G e 4G. Quando un dispositivo è connesso a due SN, ciascuna rete deve
gestire e utilizzare autonomamente un proprio contesto di sicurezza. Nel caso
in cui il dispositivo sia registrato su due SN all'interno della stessa rete
pubblica mobile terrestre (PLMN), che siano 3GPP o non-3GPP, il dispositivo
stabilisce due connessioni NAS (Non-Access Stratum) separate per ciascuna rete,
ma condivide un contesto di sicurezza NAS comune, che include un unico insieme
di chiavi e algoritmi di sicurezza.

Le procedure per mantenere o scartare un contesto di sicurezza durante la
transizione di stato dicano che la configurazione della tipologia di handover è
a discrezione dell'operatore, basandosi sui requisiti di sicurezza individuali.
Di conseguenza, la sicurezza durante l'handover diventa una funzione opzionale
e non obbligatoria, il che potrebbe portare alcuni operatori a implementare
procedure di handover potenzialmente non sicure.

La separazione crittografica e la protezione contro attacchi di replay per due
connessioni NAS attive vengono garantite attraverso un contesto di sicurezza
NAS condiviso, con parametri distinti per ciascuna connessione. Il NAS impiega
algoritmi di cifratura a 128 bit per garantire l'integrità e la riservatezza
dei dati. È importante notare, tuttavia, che sono previste anche opzioni di
cifratura e protezione dell'integrità nulle. Inoltre, se il dispositivo non
dispone di un contesto di sicurezza NAS, il messaggio NAS iniziale viene
trasmesso in chiaro, includendo l'identificatore dell'abbonato e le capacità di
sicurezza del dispositivo stesso.

Nel controllo delle risorse radio, l'integrità e la riservatezza vengono
garantite dal livello PDCP (Packet Data Convergence Protocol) che opera tra il
dispositivo e il gNB.\@ È importante notare che nessun livello al di sotto del
PDCP è soggetto a protezione dell'integrità. La protezione contro gli attacchi
di replay è attivata quando la protezione dell'integrità è in funzione, tranne
nel caso in cui sia selezionata la protezione dell'integrità nulla. I controlli
di integrità RRC vengono effettuati sia sul dispositivo che sul gNB, e se un
controllo di integrità fallisce dopo che la protezione è stata attivata, il
messaggio corrispondente viene immediatamente scartato.

Passando al piano utente, la funzione di gestione delle sessioni (SMF) si
occupa di fornire la politica di sicurezza per una sessione PDU (Protocol Data
Unit) al gNB durante la fase di stabilimento della sessione. Se la protezione
dell'integrità non è attivata per i portatori radio di dati (DRB), né il gNB né
il dispositivo saranno in grado di proteggere l'integrità del traffico di tali
DRB.\@ Allo stesso modo, se la cifratura del piano utente non è attivata per i
DRB, il traffico non verrà cifrato. La SMF locale ha la possibilità di
sovrascrivere l'opzione di riservatezza presente nella politica di sicurezza
del piano utente ricevuta dalla SMF della rete di origine (HN).

Infine, per quanto riguarda la privacy dello ID di abbonamento, il SUCI
rappresenta la versione nascosta dell'identificatore di abbonamento permanente
del 5G (SUPI). Questo viene trasmesso via etere per evitare l'esposizione
dell'identità dell'utente in chiaro. Il SUCI è generato dal SUPI utilizzando la
chiave pubblica dell'operatore e un metodo di crittografia asimmetrica
probabilistica, il quale aiuta a prevenire il tracciamento dell'identità.
Tuttavia, il sistema di protezione nulla del SUPI è utilizzato durante sessioni
d'emergenza non autenticate, se configurato dalla rete di origine (HN), oppure
quando la chiave pubblica dell'operatore non è stata fornita. Le specifiche del
5G definiscono anche un identificatore temporaneo, il 5G Globally Unique
Temporary Identifier (5G-GUTI), per ridurre l'esposizione del SUPI e del
SUCI.\@ Il 5G-GUTI deve essere riassegnato in base ai trigger del dispositivo,
ma la frequenza di tale riassegnazione è lasciata alla discrezione
dell'implementazione della rete.

\section{Attacks and Vulnerabilities}

\subsection{5G NSA}
Il \textbf{5G Non-Standalone (NSA)} è un'architettura transitoria che sfrutta
la infrastruttura esistente del \texttt{4G LTE} per la gestione della
segnalazione e del controllo della rete, utilizzando il 5G principalmente per
aumentare la capacità e la velocità di trasmissione dei dati. Sebbene questa
configurazione acceleri l'adozione del 5G, porta con sé anche una serie di
sfide e vulnerabilità di sicurezza. Infatti, le reti NSA dipendono dal core 4G
per il controllo, il che significa che molte delle vulnerabilità già presenti
nelle reti LTE possono essere trasferite nell'ambito del 5G NSA.\@ \\ Studiare
la sicurezza delle reti 5G NSA diventa quindi cruciale non solo per mitigare le
vulnerabilità già presenti, ma anche per prepararsi alla transizione verso il
5G Standalone.\@ \\ \texttt{5G NSA} è un metodo in cui la rete centrale è
configurata come un core di pacchetti basata su LTE e vengono utilizzati sia
\texttt{\hyperlink{eNB}{evolved node B (eNB)}} sia \texttt{\hyperlink{gNB}{next
		generation node B (gNB)}}. La comunicazione 5G NSA si divide in una componente
wireless e una cablata. La parte wireless comprende la rete di accesso radio
(RAN), che collega l'apparecchiatura utente (UE) alla stazione base, mentre la
parte cablata riguarda la rete centrale (CN), che connette la stazione base
alla rete di servizio. In questo contesto, si distinguono due piani: il piano
di controllo (CP), responsabile della gestione del traffico di segnalazione tra
il terminale e la rete, e il piano utente (UP), incaricato di gestire il
traffico dati tra il terminale e i servizi di rete, come Internet o le chiamate
vocali \\ Il fatto che la configurazione NSA utilizzzi una rete core bassata su
EPC (evolved Packet Core) di tipo LTE e il node eNB fa in modo che \texttt{5G
	NSA} erediti le vulnerabilità di sicurezza esistenti in \texttt{LTE}.\@ \\ Le
principali minacce alla sicurezza nelle reti \texttt{5G NSA} sono
quindi~\cite{Park20215G}:
\begin{itemize}
	\item Fuga di informazioni
	\item Attacchi DoS
	\item Intercettazione
	\item Uso non autorizzato di dati
\end{itemize}

\subsubsection{Minacce al Radio Access Network}
Le minacce alla sicurezza del Radio Access Network (RAN) si concentrano sulle
vulnerabilità della parte radio della rete, che include le stazioni base e i
dispositivi mobili. \\[0.2cm]
\textbf{Fuga di informazioni}:
Le minacce includono lo sniffing del \texttt{paging} e la decodifica
dell'\texttt{\hyperlink{IMSI}{IMSI}}.\@
Il \texttt{paging} è un processo utilizzato per inviare notifiche ai dispositivi mobili
riguardo chiamate, messaggi o altre comunicazioni. Le stazioni base trasmettono questi
messaggi in broadcast, consentendo ai dispositivi di rispondere e stabilire una connessione.
L'\texttt{\hyperlink{IMSI}{IMSI}} è un identificatore unico per un abbonato,
fondamentale per l'autenticazione nella rete.
Lo sniffing del paging consente a un attaccante
di intercettare i messaggi di paging e ottenere l'\texttt{\hyperlink{IMSI}{IMSI}} dell'utente,
utilizzando dispositivi come il Software Defined Radio (SDR).
\\[0.2cm]
\textbf{Denial of Service (DoS) per l'utente:} Questo tipo di attacco include varie forme,
come il DoS alla connessione \texttt{\hyperlink{RRC}{RRC}} (Radio Resource Control),
il DoS di rifiuto \texttt{\hyperlink{RRC}{RRC}}
e il DoS di rilascio \texttt{\hyperlink{RRC}{RRC}}.\@
Il DoS alla connessione \texttt{\hyperlink{RRC}{RRC}} è particolarmente grave, poiché sfrutta
l'\texttt{\hyperlink{S-TMSI}{S-TMSI}} (SAE-Temporary Mobile Subscriber Identity)
della vittima, precedentemente ottenuto attraverso una fuga di informazioni.
Le stazioni base non implementano procedure di autenticazione per i terminali,
consentendo agli attaccanti di interferire con l'accesso wireless della vittima.
L'attaccante può inviare un messaggio di richiesta di connessione
\texttt{\hyperlink{RRC}{RRC}} utilizzando il valore
\texttt{\hyperlink{S-TMSI}{S-TMSI}} della vittima, portando alla disconnessione della connessione
\texttt{\hyperlink{RRC}{RRC}} della vittima e stabilendo una connessione con l'attaccante,
che può poi continuare a inviare richieste per impedire l'accesso legittimo al servizio.
\\[0.2cm]
\textbf{DoS della stazione base:} Un attacco DoS può anche mirare a esaurire le risorse
della stazione base, rendendo difficile per gli utenti legittimi connettersi.
Quando un terminale cerca di stabilire una connessione, utilizza il protocollo RRC,
che prevede un processo di accesso casuale. Gli attaccanti possono sfruttare questo
processo per inviare richieste non autorizzate, aumentando il numero di connessioni
RRC attive e sovraccaricando la stazione base, causando ritardi o interruzioni nei servizi.
\\[0.2cm]
\textbf{Eavesdropping:} Sebbene le impostazioni di sicurezza AS dovrebbero prevenire
l'intercettazione, ci sono circostanze in cui ciò può avvenire. Il traffico vocale è
gestito tramite il protocollo \hyperlink{RTP}{RTP},
e il bearer vocale, a differenza dei bearer dati,è dedicato e garantisce la qualità del servizio.
Tuttavia, se un attaccante riesce a
registrare la comunicazione vocale cifrata e poi effettua una chiamata utilizzando lo
stesso ID del bearer, può estrarre il keystream necessario per decodificare le
comunicazioni precedenti.
\\[0.2cm]
\textbf{Utilizzo non autorizzato dei dati:} Nelle reti mobili esistono bearer predefiniti
e dedicati, progettati per comunicazioni legittime. Un attaccante può sfruttare questi
bearer per accedere ai dati in modo non autorizzato, ad esempio stabilendo
comunicazioni senza costi. Inoltre, il masquerading del chiamante,
noto anche come caller spoofing, è un attacco in cui l'attaccante falsifica l'identità
del chiamante, ingannando la vittima.

\subsubsection{Core Network Security Threats}
\textbf{Fuga di informazioni:} Le reti core 5G NSA possono essere suddivise in dispositivi
EPC per l'elaborazione dei dati e dispositivi IMS per i servizi.
Gli attaccanti possono colpire il protocollo GTP tra i dispositivi EPC o il protocollo SIP
nei dispositivi IMS, in base alle informazioni che desiderano ottenere. Una tecnica comune
è l'iniezione di pacchetti GTP-C per estrarre informazioni IP dai dispositivi EPC.\@
\\
\textbf{Esaurimento degli indirizzi IP:} Utilizzando la tecnica GTP-in-GTP, un attaccante
può esaurire il pool di indirizzi IP della rete inviando richieste di sessione con numeri
di terminale incrementati, impedendo così ai terminali legittimi di connettersi.
\\
\textbf{Denial of Service (DoS):} Un attacco DoS può essere eseguito inviando ripetutamente
messaggi di richiesta di connessione alla rete 5G NSA, sovraccaricando il core di rete e
causando l'interruzione del servizio.
\\
\textbf{Manipolazione del NAS:} I messaggi del protocollo NAS, utilizzati per la segnalazione
tra terminali e core di rete, non sempre sono protetti da cifratura. Un attaccante può
sfruttare questa vulnerabilità installando una stazione base malevola per manipolare i
messaggi e alterare i parametri critici per la cifratura e l'integrità dei dati.
\\
\textbf{Intercettazione:} Le comunicazioni vocali nelle reti 5G utilizzano la rete IMS e
il protocollo SIP.\@ Se l'attaccante riesce a disabilitare la cifratura IPSec nel terminale
della vittima, può intercettare il traffico vocale non criptato attraverso attacchi di
tipo man-in-the-middle (MitM).
\\
\textbf{Spoofing:} Lo spoofing di IP è un attacco comune in cui l'attaccante invia pacchetti
con indirizzi IP falsificati, causando problemi di fatturazione e potenziali attacchi DoS.
Inoltre, lo spoofing del campo \texttt{from} nei pacchetti SIP o MMS può essere utilizzato
per il voice phishing, presentando numeri falsificati sul terminale ricevente.

\subsection{Attacchi alle reti 5G SA}
Le minacce alla sicurezza del 5G SA classificate da ENISA sono anche in questo
similmente al caso del 5G NSA:\@
\begin{itemize}
	\item SPAM
	\item Spoofing degli identificatori
	\item Tracciamento della posizione
	\item DoS
	\item Frode degli abbonati
	\item Intercettazione messaggi
	\item Attacchi al routing delle chiamate
	\item Attacchi di infiltrazione
\end{itemize}

Nella Figure~\ref{fig:attacco} mostra la classificazione dei tipi di attacco
correlati ai protocolli di rete sulla rete core 5G.

\subsection{Attacchi basati sul protocollo}
Di seguito sono elencati i principali tipi di attacchi basati su protocolli
nelle reti 5G~\cite{Kim20205G}.

\subsubsection{Attacchi basati sul protocollo RRC}
Il protocollo \texttt{\hyperlink{RRC}{RRC}} regola l'istituzione, la
riconfigurazione e il rilascio delle risorse radio tra l'utente (UE) e la rete
di accesso radio. Gli attaccanti possono sfruttare vulnerabilità in questo
protocollo per eseguire attacchi come la manomissione dell'ID dell'abbonato,
attacchi DoS (Denial of Service) contro le stazioni base e bypass
dell'autenticazione dovuti alle debolezze del chipset di base (baseband)
dell'UE e dell'attrezzatura della stazione base.

\subsubsection{Attacchi basati sul protocollo NAS}
Il protocollo \texttt{\hyperlink{NAS}{NAS}}, ovvero Non Access Stratum, è
composto da messaggi di controllo che gestiscono vari aspetti cruciali nelle
reti 5G, come l'autenticazione degli utenti finali, la loro mobilità e la
posizione tra il dispositivo utente e l'equipaggiamento della rete core.\@
Questa interazione è fondamentale per garantire che gli utenti siano
correttamente autenticati e per gestire il loro spostamento all'interno della
rete. Tuttavia, i malintenzionati possono approfittare di questo protocollo per
lanciare attacchi DoS (Denial of Service) mirati allo eequipaggiamento
\texttt{\hyperlink{MME}{MME}}, il che potrebbe provocare l'interruzione dei
servizi per gli utenti. Inoltre, c'è il rischio di perdita di dati sensibili,
come le informazioni di identificazione degli abbonati, che potrebbero essere
sfruttate in vari modi, incluso l'attacco man-in-the-middle. Questo accade a
causa di vulnerabilità presenti nella specifica del protocollo stesso, così
come a causa di bypass delle misure di autenticazione e errori nella gestione
dei messaggi, che possono compromettere ulteriormente la sicurezza delle
comunicazioni nella rete.

\subsubsection{Attacchi basati sul protocollo GTP}
Il GTP (GPRS Tunneling Protocol) è un protocollo di controllo che gestisce la
creazione e il rilascio dei tunnel all'interno della rete core per la
trasmissione di dati IP.\@ Esistono messaggi GTP-C per il controllo delle
sessioni di tunneling e messaggi GTP-U per la trasmissione dei dati. A causa
della sua progettazione iniziale, il protocollo GTP non ha preso in
considerazione la sicurezza, portando a vulnerabilità sfruttabili. Ricerche
condotte da Positive Technology, GSMA e KISA hanno evidenziato la possibilità
di attacchi man-in-the-middle e DoS tramite la contraffazione dei valori dei
campi dei messaggi GTP.\@

\subsubsection{Attacchi basati sul protocollo Diameter}
Il protocollo Diameter è utilizzato per l'autenticazione, l'autorizzazione e la
contabilizzazione (AAA) ed è fondamentale per il controllo delle politiche di
qualità del servizio. Gli attacchi che sfruttano questo protocollo includono il
dirottamento delle connessioni e gli attacchi di ripetizione.

\subsubsection{Attacchi basati sul protocollo SS7}
Il protocollo SS7, sebbene originariamente progettato per 2G e 3G, continua a
rappresentare una minaccia a causa del roaming tra paesi collegati a reti
legacy. Gli attacchi possibili includono SPAM, spoofing, tracciamento della
posizione, frodi, intercettazioni e attacchi DoS. Sebbene l'attenzione si sia
concentrata principalmente sugli attacchi di protocollo nel piano di controllo,
è previsto che gli attacchi DDoS basati sui messaggi di protocollo nel piano
utente diventino un problema significativo.

\subsubsection{Attacchi nel piano utente}
Nel piano utente, i potenziali attacchi sono classificabili in tre categorie
principali:

\begin{itemize}
	\item \textbf{Attacchi basati sul protocollo GTP-U:} Questo protocollo di tunneling opera collegandosi al GTP-C del piano di controllo. Gli attacchi tipici includono DoS che caricano l'equipaggiamento core 5G e possono sfruttare le vulnerabilità del protocollo per ottenere informazioni sulla rete e sugli abbonati.

	\item \textbf{Attacchi basati sul protocollo SIP:} Utilizzato per fornire servizi VoIP su LTE, gli attaccanti possono sfruttare il protocollo SIP per eseguire attacchi DoS e di dirottamento delle chiamate attraverso messaggi come INVITE.\@

	\item \textbf{Attacchi basati su protocolli IoT:} Con l'aumento del traffico dati generato dai dispositivi IoT, sono previsti vari tipi di attacchi DDoS, mirati all'infrastruttura della rete 5G, ai server di applicazione e ai dispositivi connessi. Gli attacchi possono esaurire le risorse dell'infrastruttura, causando interruzioni su larga scala.
\end{itemize}

\begin{figure}[h]
	\centering
	\includegraphics[width=\textwidth]{ATTACCHI.png}
	\caption{Classification of 5G Threat based network protocol}\label{fig:attacco}
	\texttt{Source: 5G core network security issues and attack classification from
		network protocol perspective, 2020}
\end{figure}

%--------------------------------------------------------------------------
\section{Conclusions}

Ecco una versione migliorata del tuo testo:

La tecnologia 5G rappresenta una delle ultime frontiere nel campo delle
comunicazioni senza fili. Essa ha il potenziale di rivoluzionare molte attività
umane e il nostro modo di concepire le comunicazioni moderne, grazie a elevate
velocità di trasmissione, bassa latenza e un alto throughput. Tuttavia, con
l'avanzamento della tecnologia mobile, si sviluppano anche nuove minacce che
possono compromettere la sicurezza di queste reti.

In questo articolo, vengono analizzate le principali vulnerabilità associate al
5G. Molte di queste vulnerabilità sono ereditate dal 4G, mentre altre sono
completamente nuove. La ricerca sulla sicurezza e l'identificazione di
soluzioni innovative rappresentano un compito cruciale per tutti gli attori
coinvolti, dai fornitori di servizi di rete a coloro che definiscono i
protocolli di comunicazione. Questo diventa particolarmente necessario,
considerando che il 5G è sempre più fondamentale per settori della vita
quotidiana che possono diventare estremamente delicati se non gestiti
correttamente, come la telemedicina e i veicoli a guida autonoma.\appendix
\section{Termini da ricordare}
\begin{itemize}
	\item \textbf{\hypertarget{bandalargaMobile}{Banda Larga Mobile}}: La caratteristica principale
	      della banda larga mobile è la possibilità di fornire accesso a Internet in modalità wireless,
	      senza essere limitati a una connessione fissa, come quella via cavo o fibra ottica.

	\item \textbf{\hypertarget{MIMO}{Massive MIMO}}: I sistemi wireless Multiple Input Multiple Output
	      sfruttano più antenne di trasmissione e ricezione per aumentare la capacità di rete,
	      migliorando il throughput dei dati e servendo un maggior numero di utenti. MIMO suddivide
	      il segnale in sottosegnali a bassa velocità, trasmessi su antenne spazialmente separate
	      sullo stesso canale di frequenza. Grazie alla propagazione su percorsi multipli, il
	      ricevitore separa i segnali in flussi paralleli per recuperare il segnale originale.
	      MIMO aumenta la capacità del canale senza consumare ulteriore larghezza di banda o
	      potenza e la velocità può crescere aggiungendo più antenne. Nel 5G, la tecnologia
	      Massive MIMO va oltre la configurazione $2 \times 2$ del 4G, utilizzando numerosi
	      flussi simultanei per aumentare la capacità di rete e l'efficienza spettrale.
	      L'array di antenne più grande permette un'elaborazione coerente del segnale, adattandosi
	      velocemente ai cambiamenti del canale di propagazione~\cite{Kathavate2021Critical}.

	\item \textbf{\hypertarget{mmWave}{mmWave}}: Le comunicazioni ad onde millimetriche
	      si rifereiscono all'uso di onde elettromagnetiche molto elevate, tipicamente comprese
	      tra \textit{30 GHz e 300GHz}. Questo onde sono chiamate millimetriche perchè la loro
	      lunghezza d'nda varia tra 1mm e 30mm, che sono molto più corte rispetto alle ond radio
	      tradizionalmente usate. Esse permetono velocità molto elevate e basssa latenza, ma hanno
	      una portata limitata e scarsa capacità di penetrazione, richiedendo infrastrutture dense
	      come small cells e tecnologie avanzate come il beamforming. Utilizzate insieme
	      a frequenze più basse per garantire una copertura completa,
	      le mmWave sono cruciali per migliorare la capacità delle
	      reti in aree ad alta densità di utenti.

	\item \textbf{\hypertarget{EAP-AKA}{EAP-AKA}}: è un protocollo di autenticazione
	      progettato per consentire l'autenticazione sicura tra un dispositivo mobile e una rete,
	      basato sul concetto di chiavi simmetriche.
	      Quando un dispositivo, desidera connettersi a una rete,
	      invia una richiesta di autenticazione utilizzando un \texttt{SUCI},
	      una versione cifrata del suo identificatore permanente, il \texttt{SUPI},
	      consentendo alla rete di identificare il dispositivo senza rivelare
	      l’identità reale dell’utente e garantendo la sua privacy.
	      La rete, ricevuta la richiesta, utilizza il SUCI per accedere alle credenziali
	      memorizzate nel database dell’\texttt{ARPF}, dove è custodita la chiave segreta \texttt{K},
	      condivisa tra il dispositivo e la rete; se viene scelto il protocollo \texttt{EAP-AKA},
	      la rete avvia il processo di autenticazione generando una \texttt{sfida}
	      per il dispositivo composta da tre elementi: un numero casuale \texttt{(RAND)},
	      un messaggio di autenticazione \texttt{(AUTN)} e la risposta attesa \texttt{(XRES)}, che vengono
	      inviati al dispositivo. Questo verifica il valore \texttt{AUTN} utilizzando la chiave \texttt{K} nella sua \texttt{USIM},
	      e, se l'\texttt{AUTN} è valido, calcola una risposta \texttt{(RES)} basata su \texttt{RAND} e la chiave \texttt{K},
	      quindi la invia alla rete, che confronta la \texttt{RES} ricevuta con la \texttt{XRES} generata in precedenza;
	      se corrispondono, l’autenticazione ha successo e entrambe le parti hanno verificato
	      l'identità reciproca. Successivamente, si genera la chiave di sessione,
	      derivata dalla chiave \texttt{K} e dai materiali generati nella sfida,
	      con chiavi come \texttt{CK} (per la cifratura dei dati) e \texttt{IK} (per l’integrità dei messaggi),
	      essenziali per proteggere le comunicazioni. Una volta completato il processo di
	      autenticazione e derivate le chiavi di sessione, il dispositivo e la rete possono
	      iniziare a scambiare dati in modo sicuro, sapendo che le comunicazioni sono protette
	      da solide misure di sicurezza.

	\item \textbf{\hypertarget{5G AKA}{5G AKA}}: Il processo \texttt{5G AKA} inizia con l'invio,
	      da parte del dispositivo, di una richiesta di autenticazione alla rete tramite il \texttt{SUCI},
	      una versione cifrata del \texttt{SUPI}, che protegge l'identità dell'utente durante la registrazione.
	      La rete riceve questa richiesta e la invia al database delle credenziali,
	      che contiene la chiave segreta \texttt{K} condivisa tra il dispositivo e la rete.
	      Se viene scelto il protocollo \texttt{5G AKA}, la rete genera una \texttt{sfida}
	      composta da un numero casuale \texttt{(RAND)}, un'autenticazione temporanea \texttt{(AUTN)}
	      e una risposta attesa \texttt{(XRES)}, che vengono inviati al dispositivo.
	      Il dispositivo, usando la chiave \texttt{K} memorizzata nella \texttt{USIM},
	      verifica l'\texttt{AUTN} per confermare l'identità della rete;
	      se la verifica è valida, calcola una risposta \texttt{(RES)} basata su \texttt{RAND} e \texttt{K},
	      che viene poi inviata alla rete. La rete confronta la \texttt{RES} con la \texttt{XRES} generata in precedenza,
	      e se corrispondono, l'autenticazione ha successo.\@
	      A questo punto, vengono generate le chiavi di sessione per la cifratura
	      e l'integrità delle comunicazioni, garantendo la sicurezza dei dati trasmessi.
	      Una caratteristica importante del \texttt{5G AKA} è il rafforzamento della privacy dell'utente,
	      grazie alla separazione delle chiavi di cifratura e integrità e a meccanismi più robusti
	      per prevenire attacchi di tracciamento e correlazione delle identità,
	      migliorando significativamente la sicurezza rispetto alle generazioni precedenti.\@

	\item \textbf{FBMC}\hypertarget{FBMC}{}: Il Filter Bank Multicarrier (FBMC) è una tecnica di
	      modulazione che divide un segnale in più sottocanali, applicando filtri per ridurre
	      le interferenze tra questi, migliorando così l'efficienza spettrale rispetto all'OFDM.\@

	\item \textbf{FullDuplex}\hypertarget{FullDuplex}{}:Il full duplex è una modalità di comunicazione
	      in cui i dati possono essere trasmessi e ricevuti contemporaneamente tra due dispositivi
	      o punti. In altre parole, entrambe le parti possono inviare e ricevere informazioni allo
	      stesso tempo, senza dover aspettare che una delle due abbia finito di trasmettere.

	\item \textbf{Ultra Dense Networking}\hypertarget{UDN}{}:L'Ultra Dense Networking (UDN) è una
	      architettura di rete progettata per migliorare la capacità e la copertura della rete
	      in ambienti ad alta densità di utenti o dispositivi.DN si basa sull'idea di aumentare
	      il numero di celle o piccole stazioni base (small cells) in un'area geografica,
	      riducendo la distanza tra queste stazioni e i dispositivi connessi. Questo riduce il
	      carico su ogni singola stazione base, migliorando la capacità di banda e la qualità
	      del segnale.

	\item \textbf{Software-Defined Networking}\hypertarget{SDN}{}:  Software-Defined Networking (SDN)
	      è un approccio alla gestione delle reti che separa il piano di controllo (control plane)
	      dal piano dati (data plane). Tradizionalmente, i router e gli switch svolgono sia il
	      compito di instradare il traffico (piano dati) che di decidere come farlo
	      (piano di controllo). SDN sposta il piano di controllo in un'entità software centrale
	      chiamata \texttt{controller}, che ha una visione globale della rete e può programmare
	      dinamicamente come i pacchetti devono essere gestiti dagli switch, semplificando la
	      gestione della rete e migliorandone l'agilità.

	\item \textbf{Network Function Virtualization}\hypertarget{NFV}{}: La Network Function
	      Virtualization (NFV) è una tecnologia che virtualizza le funzioni di rete, come firewall,
	      router, load balancer e altri dispositivi di rete, su server standard,
	      eliminando la necessità di hardware specializzato. NFV consente di distribuire e gestire
	      le funzioni di rete come software, migliorando la scalabilità, la velocità di
	      implementazione e riducendo i costi.

	\item \textbf{\hypertarget{S-TMSI}{S-TMSI}}: è un identificatore temporaneo
	      utilizzato nelle reti 4G LTE e 5G per rappresentare un abbonato senza rivelarne
	      l'IMSI (International Mobile Subscriber Identity).
	      Esso viene generato dalla rete e cambia periodicamente,
	      riducendo il rischio di tracciamento e attacchi informatici.

	\item \textbf{\hypertarget{RRC}{RRC}}: RRC è un protocollo fondamentale
	      nel piano di controllo delle reti mobili ed è responsabile della gesione delle
	      connessioni radio tra l'utente e la rete. Le principali funzioni
	      del protocllo RRC includono: l'instaurazione e il rilascio della connesione radio,
	      la gestione della mobilità, il controllo qalità  e la trasmissione di informazioni
	      di confgurazione e segnalazione.

	\item \textbf{eNB}\hypertarget{eNB}{}: Rappresenta il  nodo di accesso radio delle reti
	      \texttt{4G LTE}.\@ eNB è responsabile della trasmissione e ricezione dei segnali radio
	      tra l'utente finale e la rete core \texttt{LTE} e ha funzioni di gestione delle risorse radio,
	      codifica e decodifica, controllo di potenza, gestione della mobilità e dellos cheduling dei dati.

	\item \textbf{gNB}\hypertarget{gNB}{}: Il next generation Node B è l'elemento
	      della rete di accesso radio nella tecnologia 5G.
	      Oltre a supportare i tradizionali servizi a banda larga, il gNodeB consente
	      funzionalità avanzatate come le comunicazioni massive per dispositive IoT e La
	      gestione del traffico con qualità del servizio variabile in base alle
	      esigenze dell'applicazione.

	\item \hypertarget{IMSI}{\textbf{IMSI}}: International Mobile Subscriber Identity
	      è un numero identificativo univoco,
	      composto da 15 cifre associato a ciascun abbonato nella rete mobili.
	      Questo numero è utilizzato per identificare e autenticare un utente all'interno
	      di una rete mobile.\@ \texttt{IMSI} è memorizzato nella scheda SIM e include il
	      codice del paese \texttt{(MCC)},
	      il codice dell'operatore \texttt{(MNC)}
	      e un numero identificativo dell'abbbonato \texttt{(MSIN)}.

	\item \textbf{\hypertarget{RTP}{RTP}}: Il Real-time Transport Protocol (RTP) è un protocollo
	      di rete utilizzato per la trasmissione di dati in tempo reale, come voce e video, su reti IP.\@

	\item \textbf{\hypertarget{NAS}{NAS}}: Gestisce la comunicazione tra il dispositivo dell'utente e il core network,
	      occupandosi di autenticazione e registrazione,
	      assicurando che solo i dispositivi autorizzati accedano alla rete.
	      Inoltre, si occupa della mobilità, consentendo agli utenti di passare da una
	      cella all'altra senza interruzioni. Gestisce anche le sessioni di dati,
	      stabilendo e terminando le connessioni e mantenendo la qualità del servizio.

	\item \textbf{\hypertarget{MME}{MME}}: L'Mobility Management Entity (MME)
	      è un componente software nel core network delle reti mobili 4G e 5G, responsabile della
	      gestione della mobilità degli utenti e della segnalazione.
	      Svolge funzioni chiave come l'autenticazione degli utenti, la gestione delle sessioni,
	      il monitoraggio della posizione e l'instradamento delle richieste di servizio verso
	      altre entità di rete. L'MME comunica con dispositivi terminali e altri elementi
	      della rete utilizzando protocolli come NAS (Non Access Stratum) e
	      GTP (GPRS Tunneling Protocol).
\end{itemize}
\clearpage
\printbibliography\end{document}
